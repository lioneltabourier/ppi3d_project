\documentclass{article}

% Language setting
% Replace `english' with e.g. `spanish' to change the document language
\usepackage[english]{babel}

% Set page size and margins
% Replace `letterpaper' with `a4paper' for UK/EU standard size
\usepackage[letterpaper,top=2cm,bottom=2cm,left=3cm,right=3cm,marginparwidth=1.75cm]{geometry}

% Useful packages
\usepackage{amsmath}
\usepackage{amssymb}
\usepackage{graphicx}
\usepackage[colorlinks=true, allcolors=blue]{hyperref}

\graphicspath{{./images/}}

\newcommand{\mytodo}{ \begin{center} \textcolor{red}{[TODO]} \end{center}}



%%%%%%%%%%%%%

\title{Working document:\\
structural link prediction using PPI3D dataset}
\author{}
\date{}

\begin{document}
\maketitle

%\begin{abstract}
%Your abstract.
%\end{abstract}

%%%%%%%%%%%%%

\section{Data and preprocessing}

Starting from dataset \texttt{clustering95/human\_9606/filtered\_human\_area\_less12000.csv}

\subsection{Preprocessing}

\begin{itemize}
    \item the edges considered are limited to protein-protein interactions (exclude protein-peptide)
    \item we exclude \texttt{homo} edges (from the protein to itself)
    \item we consider that \texttt{s1\_seq\_cluster\_95} can be used as a proxy to identify a given protein
\end{itemize}

\subsection{Characteristics of the PPI network}

$n= 10,077$ proteins,  $ m= 12,957 $ distinct interactions 
%
$ \Rightarrow  $ average degree $ \overline{d} = \frac{2m}{n} \simeq 2.57 $
%
and density $ \delta = \frac{2m}{n(n-1)} \simeq 2.55 \times 10^{-4}$

\medskip

For comparison:
%
\begin{itemize}
    \item[-] \texttt{STRING} (confidence $\geq 900$): $n= 6,743$,  $ m= 67,730$
    %
    \item[-]  \texttt{Heine et al.} $n= 5,457$,  $ m=28,779 $
    %
    \item[-]  \texttt{Lit-bm-13} $n= 3,391$,  $ m= 4,905 $
    %
    \item[-]  \texttt{Lit-nb-13} $n= 5,545$,  $ m= 11,044 $
\end{itemize}


%%%%%%%%%%%%%

\section{Basic predictions on the PPI dataset}

We implement basic predictors: \texttt{L3} and variants, and \texttt{Adamic-Adar} for comparison in an unsupervised way to the data. 
%
The related Figure is~\ref{fig:res-ppi}.

\begin{figure}
    \centering
    \includegraphics[width=0.45\linewidth]{ppi3d_c95_human_net_75-25.pdf}
    \includegraphics[width=0.45\linewidth]{ppi3d_c95_human_net_50-50.pdf}
    \caption{Left: data split (train/test) is 75-25; right is 50-50 (Yuen data split).}
    \label{fig:res-ppi}
\end{figure}

\medskip

Comments:
\begin{itemize}
    \item the performances are not quite similar to what we find on other relatively small datasets (\texttt{Heine et al.}, \texttt{Lit-bm-13}, \texttt{Lit-nb-13}), see~\cite{yuen2023normalized}: L3 and variants are less efficient than usual, especially at larger recall 
    $ \Rightarrow $ we may have performance improvement when stacking
\end{itemize}
%
\mytodo

%%%%%%%%%%%%%

\section{Making a binding-site network}

Based on the idea that the jigsaw puzzle hypothesis should work better at the scale of the binding site, we transform the network into a binding site network (BSN).
%
For this purpose we use \texttt{s1\_binding\_site\_cluster\_95} as the identifier of the nodes.

\medskip

Characteristics of the BSN:  
%
$n= 28,945$ distinct binding sites,  $ m= 17,714 $ interactions 
%
$ \Rightarrow  $ average degree $ \overline{d} = \frac{2m}{n} \simeq 1.23 $
%
and density $ \delta = \frac{2m}{n(n-1)} \simeq 4.23 \times 10^{-5}$.
%
It is thus very sparse, making prediction on it is hardly doable.

Justas suggests to use a lower binding site similarity cluster (70 or 40), which would make it denser.
%
\mytodo

%%%%%%%%%%%%%

\section{Adding biology to the L3-prediction}

\subsection{Testing the jigsaw puzzle assumption}

According to the puzzle assumption, if we observe the following edges:
\texttt{x-u}, \texttt{u-v}, \texttt{v-y}, then \texttt{y-x} should be present in the network.
%
In other words, if we see a path-3 pattern in the network, we should observe that this path is closed in a length 4 cycle.

We measure in the PPI network how frequent this pattern actually is.
%
We count 147,727 path-3 patterns and 19,557 cycles of length 4.
%
As there are 4 such paths per cycle, we can count a normalized ratio (let's call it 4-transitivity), which is the probability for a 3-path to be completed in a 4 cycle in the dataset:

$$ \frac{4.\square}{\sqcap} = 0.530 $$

For comparison, if edges were totally random, then it should be equal to the density $ \delta \simeq 2.55 \times 10^{-4}$.
%
This is thus very high.

We can try to define a transitivity-based prediction to see if it can challenge L3 performances.
%
\mytodo

Also, when listing theses squares, we observe many structure like this:\\
%
\texttt{
85339 85346 85340 85351\\
85339 85346 85348 85351\\
85340 85346 85348 85351\\
etc.\\}
%
which seems to indicate that many cycles are part of a larger structure, which would appear as a larger quasi-clique.
%
It led to the assumption in subsection~\ref{subsection:superstructure}.


\subsection{Nature of the edges predicted by L3: superstructures?}
\label{subsection:superstructure}

Based on this naked-eye observation, Justas suggested that  the 4 edges cyclic structures are actually part of the same ``superstructure''.

Direct test: we measure if the edges \texttt{y-x} are actually part of the same superstructure.

In most case, we can assume that the interactions share the same pdb identifier \texttt{pdbid}.
%
We measure in the dataset if it is indeed the case in the cycles formerly enumerated.
%
I did that for the 19,557 cycles, and I find:
%
\begin{itemize}
    \item 2,708 cycles where all \texttt{pdbid} are identical (of the form \texttt{9cwt 9cwt 9cwt 9cwt})
    \item 6,574 cycles where there are 2 different \texttt{pdbid} (of the form \texttt{9cq4 9cq4 8za6 8za6})
    \item 6,535 cycles where there are 3 different \texttt{pdbid} (of the form \texttt{9cn3 3j7y 7o9m 7o9m})
    \item 3,740 cycles where all \texttt{pdbid} are different (of the form \texttt{7w3m 9m2w 8usb 9e8o})
\end{itemize}

In my opinion, it rather goes in the sense of Justas' assumption that in most cases, we have superstructures containing 4-cycles and that's what L3 would mostly predict.


Also, it is an underestimate because the database is made in a way that might lead to a different  \texttt{pdbid} even if the interactions are involved in the same structure.
%
The reason is that the  \texttt{pdbid} selected is the ``best'' one, and if the PPI is involved in several structures, it might not be the one corresponding to the superstructure that we are looking for.
%
I should go back to the original dataset to look for the different \texttt{pdbid} involving the edges. 
%
\mytodo


\subsection{Nature of the edges predicted by L3: interactions involving unstructured parts of the proteins?}

Another possibility could be that the edges predicted by L3 correspond to interactions involving unstructured parts of the protein.

One possibility could be to use AlphaFold to give an estimate of the fraction of the protein that is unstructured to see if it is the case.

I don't know how to address that precisely for the moment.
%
\mytodo

\section{Complementarity measures and L3 prediction}

I want to know if the L3 predicted edges (or any measure predicted edges actually) correspond to more complementary proteins.

This suppose to define some notion of complementarity, itself based on the notion of similarity.
%
We could use 
\texttt{s1\_seq\_cluster\_70} or \texttt{s1\_seq\_cluster\_40} as an intermediate to compute the notion of similarity between proteins.


For the moment I did not really address this topic.
%
\mytodo

%%%%%%%%%%%%%

%%%%%%%%%%%%%

\bibliographystyle{alpha}
\bibliography{sample}

\end{document}
